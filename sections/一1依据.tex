
\subsection{研究意义}
波动方程是二阶双曲型偏微分方程,可用于描述地震波、水波、电磁波等自然界中的波动现象,广泛的应用于无损检测、地震预测、生物医学成像等各个工程领域。
发展波动方程的高精度分析方法将有助于提升如无损检测准确率、医学成像精度等。
由于实际工程问题中通常涉及复杂几何区域、材料非均质等一系列问题,采用理论分析手段研究波动问题难以获得解析解。于是,数值仿真分析成为了研究波动方程的重要工具。
在波动问题的主流数值分析方法中,时间域和空间域将分别离散为有限个分布节点用于近似位移变量,其中空间域通常采用基于变分原理的弱形式进行离散分析,如有限元法等。
在变分原理的加持下,基于弱形式型的方法具有求解精度高、稳定性强的特点,适用于复杂空间几何构型。
而时间域则采用基于微分方程的强形式进行离散分析,如有限差分法等。
该类方法通常以迭代方式进行程序实现,程序结构简单高效。
% 但基于强形式型的方法计算误差将随着迭代步的增加而累积,时域末端计算误差大。
如果时间域和空间域均采用基于弱形式型的方法进行求解,不仅能,提高计算精度。
并且可以实现时空区域协同自适应节点加密,对整体进行并行求解,提升计算效率。
在上世纪九十年代,Hughes\cite{hughes1988}首次采用时空混合离散方案结合间断伽辽金法求解波动问题。
但在考虑该方法求解的稳定性后
未能使
,为能适用于任意节点分布情况,该问题导致时空混合离散方案未能成为分析波动方程的主流方法,阻碍了时空混合离散伽辽金法的发展。

同时,时空域分别独立离散也不利于自适应节点加密和全域并行计算。
发展适用于任意节点分布的时空混合离散伽辽金法……

第一

第二

第三

本项目

\subsection{国内外研究现状及发展动态}

\vspace{-5pt}

\begin{REF}
	\subsection*{参考文献}
	\vspace{-50pt}
	\bibliographystyle{gbt7714-2005-numerical}
	\bibliography{ref}%参考文献
\end{REF}

\newpage%自己判断是否需要
