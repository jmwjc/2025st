\vspace{-5pt}
\subsection{研究内容}
本项目将围绕波动方程时空混合离散伽辽金法存在的问题,
研究时域末端虚位移本质边界条件施加方案,
构建可缓解数值色散问题的时空混合离散无网格近似方案,
并引入变分一致型伽辽金无网格数值积分法,
建立适用于任意节点分布的时空混合离散伽辽金无网格分析方法。
具体的研究内容如下:

\subsubsection*{\bfseries (1)适用于任意节点分布的时域末端虚位移本质边界条件施加方案}
在波动方程哈密顿原理的基础上,分析能量泛函中时域末端虚位移本质边界条件下位移变量的求解空间。
将传统能量泛函中的虚位移作为拉格朗日乘子,设计全新拉格朗日乘子型能量泛函。
分析新建立的能量泛函中,经过拉格朗日乘子法投影后的位移变量空间与原空间的等价性。
同时在全新的能量泛函中引入应力边界条件与初始时刻动量边界条件,验证相对应拉格朗日乘子型伽辽金弱形式的变分一致性,证明其与欧拉--拉格朗日方程的等价性。


\subsubsection*{\bfseries (2)可缓解数值色散问题的时空混合离散再生核无网格近似方案}
探究拉格朗日乘子型能量泛函的稳定性条件,构建考虑时空混合离散阶次的局部截断误差估计表达式。
研究考虑时间维度的再生核无网格近似框架,以时空混合离散截断误差估计为基础,确定误差稳定时的再生核近似基向量中时间维度相关单项式。
根据确定基向量表达式,调整再生核近似中核函数影响域大小和表达式,以确保再生核近似中矩量矩阵可逆。
最后,通过数值算例验证所提无网格近似方案能否缓解数值色散问题。

\subsubsection*{\bfseries (3)时空混合离散下变分一致型伽辽金无网格数值积分方案}
根据时空混合离散无网格近似中基向量的元素,确定波动问题拉格朗日乘子型伽辽金弱形式的积分约束条件。
以再生光滑梯度理论框架为基础,构建时空混合离散下满足积分约束条件的变分一致型伽辽金无网格数值积分方法。
在四维情况下,引入四面体柱或六面体柱单元作为背景积分域。通过再生光滑梯度的一致性条件,优化全局数值积分点数量,确定积分点位置和权重。
最后,通过时空混合离散分片试验,测试所提伽辽金无网格数值积分方案的变分一致性和计算精度。

\subsubsection*{\bfseries (4)适用于任意节点分布的时空混合离散高效伽辽金无网格分析方法}
将时空混合离散再生核无网格近似引入拉格朗日乘子型伽辽金弱形式,结合变分一致型伽辽金无网格数值积分方案,建立波动方程时空混合离散控制方程。
进一步分析离散控制方程表达式,利用离散控制方程刚度矩阵的相似性,优化程序结构,降低内存开销。
同时,基于位移解的梯度变化,制定相对应的无网格自适应节点加密方案。
在求解离散控制方程时引入成熟并行计算库,实现时间域和空间域同时并行计算,
建立适用于任意节点分布的时空混合离散高效伽辽金无网格分析方法及其通用数值仿真程序。通过典型波动问题和实际工程算例验证所提方法的有效性和可靠性。

\subsection{研究目标}
本项目完成上述各项研究内容旨在建立适用于任意节点分布的时空混合离散伽辽金无网格分析方法,该方法能适用于任意布置的节点离散情况,缓解数值色散问题,
提升分析波动问题时的计算效率和稳定性。具体的研究目标包括:
\begin{enumerate}[label={\rmfamily (\theenumi)},left=24pt]
    \item 建立适用于任意节点离散的时域末端虚位移本质边界条件施加方案,提升时空混合离散伽辽金法对节点离散的鲁棒性;
    \item 提出适用于波动方程再生核无网格近似方案,缓解波动问题时空混合离散伽辽金无网格法的色散问题;
    \item 设计适用于高维波动方程的变分一致型伽辽金无网格法数值积分方案,降低高维背景积分域划分难度,提升计算效率;
    \item 发展任意节点分布的稳定时空混合离散高效伽辽金无网格分析方法,为实际波动问题提供可靠、稳定的数值工具。
\end{enumerate}

\subsection{拟解决的关键科学问题}

本项目拟解决的关键科学问题如下:

\subsubsection*{\bfseries (1)如何构建与哈密顿原理等价的时域末端虚位移本质边界条件}
哈密顿原理要求时域末端虚位移等于零,才可使相对应的弱形式等价于欧拉--拉格朗日方程。
由于在实际问题中,时域末端边界条件通常是未知的。单独令时域末端虚位移为零将导致整体弱形式丧失变分一致性,计算结果将出现与实际不符的情况。
目前已有方法通常是采用间断伽辽金格式规避时域末端边界,但需要采用分块网格技术提高求解的稳定性。
因此,如何正确施加时域末端虚位移本质边界条件,使时空混合离散伽辽金法适用于任意节点离散情况,是本项目研究内容(1)的关键科学问题。

\subsubsection*{\bfseries (2)如何确定时空混合无网格近似基向量阶次对求解稳定性的影响机理}
波动方程属于二阶双曲型偏微分方程,求解过程中时域离散节点间距过大将引起数值色散问题,导致计算结果发散。
传统方法可减小离散节点间距缓解该问题,随之将伴随自由度的增加所引起的计算量增大,不适合任意节点离散情况。
增加稳定项也可提升计算的稳定性,但节点间距过小时该方法将出现数值耗散问题,也不适用于任意节点离散情况。
本项目的研究内容(2)旨在利用无网格构造高阶再生特性形函数的便利性,缓解数值色散问题。因此,确定时空混合离散无网格近似基向量阶次对求解稳定性的影响机理,是该研究内容的关键科学问题。

\subsubsection*{\bfseries (3)如何优化时空混合离散变分一致伽辽金无网格数值积分方案}
由于无网格形函数通常为有理式,伽辽金无网格法需要采用变分一致型数值积分方案以保证计算精度。
变分一致型数值积分方案可通过形函数导数的一致性条件,优化全局数值积分点个数。
最终的数值积分点个数将直接影响时空混合离散伽辽金无网格法的计算效率。
相较于传统时间域与空间域分别离散的方法,时空混合离散将增加时间维度的离散,离散维度比传统方式增加了一个维度。
采用变分一致型伽辽金无网格数值积分方案时,需要额外考虑与时间维度相关的单项式积分约束条件与一致性条件。然而,为了缓解数值色散问题,时间维度相关的单项式阶次将不同于空间维度相关的单项式阶次。
如何在时空混合离散情况下优化变分一致型伽辽金无网格数值积分方案,提升整体计算效率,是本项目研究内容(3)的关键科学问题。

\newpage