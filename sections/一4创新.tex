
本项目的特色与创新之处包括以下三点:

\begin{enumerate}[label=(\theenumi),left=24pt]
    \item 时域末端虚位移本质边界条件施加方案
    实现了基于弱形式施加虚位移本质边界条件,无需分块网格划分,即可保证求解的稳定性。同时始末节点也无需匹配个数,降低节点离散的要求。
    \item 时空混合离散无网格近似方案利用再生核近似构造高阶形函数的便利性,在任意节点离散情况下即可缓解数值色散问题。基于节点离散的再生核近似局部节点加密过程实现简单,无需考虑几何拓扑关系。
    \item 在时域末端虚位移本质边界条件施加方案、时空混合离散无网格近似方案的协同作用下,时空混合伽辽金无网格分析方法可适用于任意节点离散情况,并且能轻松地进行局部区域的节点加密,更好捕捉波动问题的局部特征。
\end{enumerate}

\vspace{24pt}
