
\subsubsection*{\bfseries (1)前一个已资助期满的科学基金项目(项目名称及批准号)完成情况}
\noindent
项目批准号:12102138

\noindent
项目名称:体积不可压问题的内禀最优约束比高效无网格法

\noindent
资助类别:青年科学基金项目(C类)[原青年科学基金项目]

\noindent
执行年限:2022.01--2024.12

项目负责人按照项目计划书要求开展研究,完成预期研究目标。发表标注基金资助的论文5篇,其中SCI收录3篇,均为中科院SCI二区期刊;EI收录1篇;培养毕业研究生1名,并有4名研究生在读。

\subsubsection*{\bfseries (2)前一个已资助期满的科学基金项目后续研究进展}
项目针对体积不可压问题提出了免自锁的有限元无网格混合离散分析方法,项目结题后的研究成果将应用到不可压流体、水凝胶等体积不可压问题分析中。
同时,所提理论框架将推广至中厚板问题缓解剪切自锁现象。

\subsubsection*{\bfseries (3)前一个已资助期满的科学基金项目与本申请项目的关系}
前一个资助期满项目所提的变分一致型伽辽金无网格法本质边界条件施加方案,是本项目研究内容的理论基础之一。
该项目其余部分的研究内容与本项目无直接关系。

\subsubsection*{\bfseries (4)前一个已资助期满的科学基金项目研究工作总结摘要(限500字)}

项目在位移-压力混合离散的框架下,提出了LBB稳定性条件中稳定系数的泛函估计。
首次确定了满足LBB稳定性条件下最优体积约束比例,完善了体积约束比的理论基础。
根据LBB稳定系数估计,即可由数值方法的体积约束比判断其否满足LBB稳定性条件。
随后,项目提出了有限元无网格混合离散分析方法,依据再生核无网格近似离散的便利性,体积约束比可任意进行调整,利用该方法验证了所提LBB稳定系数估计的正确性。
当约束比调整至最优时,所提方法可保证计算精度和理论误差收敛率,消除体积自锁现象。
与此同时,项目还依托所提位移-压力混合离散框架,提出基于Hellinger-Reissner原理和Hu–Washizu原理的变分一致型伽辽金无网格分析方法,其中位移采用再生核无网格近似离散,而应力在每个积分域中假设为分片多项式。
该方法满足全域变分一致性,可保证计算精度。
且弱形式中自动包含本质边界条件施加过程,无需额外稳定项即可满足正定性条件。
所提方法完备了变分一致型无网格法的变分理论基础,也为伽辽金无网格法提供了一种变分一致且自稳定的本质边界条件施加方案。

\subsubsection*{\bfseries (5)前一个已资助期满的科学基金项目相关成果详细目录}

\noindent\textcircled{\textbf{\small 1}}
\textbf{期刊论文}

\vspace{-50pt}
\begin{thebibliography}{1}
	\bibitem[Wu et~al.(2024)Wu, Xu, Xu, and Basha]{hw2024_2}
	\textul{\textbf{Wu J}}, Xu Y, Xu B, Basha S~H.
	\newblock Quasi-consistent efficient meshfree thin shell formulation with
	  naturally stabilized enforced essential boundary conditions.
	\newblock \emph{Engineering Analysis with Boundary Elements}, 2024, 163:
	  641-655.

	\bibitem[Wu et~al.(2023)Wu, Wu, Zhao, and Wang]{hr2023_2}
	\textul{\textbf{Wu J}}, Wu X, Zhao Y, Wang D.
	\newblock A rotation-free {{Hellinger-Reissner}} meshfree thin plate
	  formulation naturally accommodating essential boundary conditions.
	\newblock \emph{Engineering Analysis with Boundary Elements}, 2023, 154:
	  122-140.

	\bibitem[吴俊超\ 等(2022)吴俊超, 吴新瑜, 赵珧冰, and
	  王东东]{hr2022_2}
	\textbf{\textul{吴俊超}}, 吴新瑜, 赵珧冰, 王东东.
	\newblock
	  {基于赫林格-赖斯纳变分原理的一致高效无网格本质边界条件施加方法}.
	\newblock 力学学报, 2022, 54: 3283-3296.

	\bibitem[Du et~al.(2022)Du, Wu, Wang, and Chen]{RKGSIgradient2022_2}
	Du H, \textul{\textbf{Wu J}}, Wang D, Chen J.
	\newblock A unified reproducing kernel gradient smoothing {{Galerkin}} meshfree
	  approach to strain gradient elasticity.
	\newblock \emph{Computational Mechanics}, 2022, 70: 73-100.

	\bibitem[付赛赛\ 等(2022)付赛赛, 邓立克, 吴俊超, 王东东, and
	  张灿辉]{RKGSIdynamic2022_2}
	付赛赛, 邓立克, \textbf{\textul{吴俊超}}, 王东东, 张灿辉.
	\newblock 再生光滑梯度无网格法动力特性研究.
	\newblock 应用力学学报, 2022, 39: 1065-1075.

\end{thebibliography}

\noindent\textcircled{\textbf{\small 2}}
\textbf{会议报告}

\begin{enumerate}[label = {[\arabic*]},leftmargin=19pt,itemsep=5pt plus 0.3ex]
	\zhkai\ensong\fontsize{12pt}{22pt}\selectfont%
	\setlength{\baselineskip}{15pt}
    \item \textbf{\textul{吴俊超}}; 自稳定免体积自锁有限元无网格混合离散伽辽金法, 第四届“计算力学与工程”学术论坛, 江苏南京,2024-12-13至2024-12-15.
    \item \textbf{\textul{吴俊超}}; 内禀最优约束比的有限元无网格混合离散分析方法, 第四届无网格粒子类方法进展与应用研讨会, 新疆乌鲁木齐, 2024-8-14至2024-8-17.
    \item \textbf{\textul{Wu Junchao}}; A consistent and naturally stabilized method for imposing meshfree essential boundary condition via Hellinger-Reissner variational principle, The 5th International Conference on Modeling in Mechanics and Materials, 广西南宁, 2023-12-8至2023-12-10.
    \item \textbf{\textul{Wu Junchao}}; A Hellinger-Reissner meshfree Kirchhoff plate formulation with naturally accommodating essential boundary conditions, 第7届亚太国际工程计算方法学术会议暨第13届全国工程计算方法学术年会, 福建厦门, 2023-11-2至2023-11-5.
    \item \textbf{\textul{吴俊超}}; 薄板问题的Hellinger-Reissner变分一致 无网格本质边界条件施加方法, 中国计算力学大会2023, 辽宁大连, 2023-8-20至2023-8-23.
    \item \textbf{\textul{吴俊超}}; 基于 Hellinger-Reissner 原理的变分一致无网格法及其本质边界条件施加方案, 第三届无网格粒子类方法进展与应用研讨会, 广西南宁, 2022-8-20至2022-8-22.
    \item \textbf{\textul{吴俊超}}; 基于 Hellinger-Reissner 原理的变分一致无网格本质边界条件施加方案, 中国力学大会2021+1, 线上, 2022-11-5至2022-11-10.
\end{enumerate}

\noindent\textcircled{\textbf{\small 3}}
\textbf{人才培养}

依托此项目,项目负责人吴俊超于 2024 年 12 月晋升副教授。项目执行期协助培养硕士研究生 6 人,其中已毕业 1 人,正在攻读硕士学位 5 人(1 人预计2025 年毕业,2 人预计 2026 年毕业,2 人预计 2027 年毕业)。其中,已毕业硕士研究生吴新瑜在读期间发表学术论文 2 篇,SCI 收录论文 1 篇、EI 收录论文 1 篇,毕业论文标注本项目批准号。

